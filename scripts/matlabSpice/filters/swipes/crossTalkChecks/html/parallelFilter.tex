
% This LaTeX was auto-generated from MATLAB code.
% To make changes, update the MATLAB code and republish this document.

\documentclass{article}
\usepackage{graphicx}
\usepackage{color}

\sloppy
\definecolor{lightgray}{gray}{0.5}
\setlength{\parindent}{0pt}

\begin{document}

    
    
\subsection*{Contents}

\begin{itemize}
\setlength{\itemsep}{-1ex}
   \item First steps
   \item nfilter3.m. Uses Straight forward theory and design specs
   \item params
   \item Spec
   \item Design constraints:
   \item Function's Inputs
   \item Function's Outputs:
   \item Next steps
\end{itemize}
\begin{verbatim}
clear
\end{verbatim}


\subsection*{First steps}

\begin{itemize}
\setlength{\itemsep}{-1ex}
   \item define spec values and calculated/rediresd param-values in this file "parallelFiler.m
   \item Run the file.
   \item This script will calculate/define the design spec for the desired filter.
   \item Then will call "nfilter3.m" to calculate the structural params of the device using the stirght forward theory.
\end{itemize}
\begin{verbatim}
lam1 =1530*(10^(-9)); % Wavelength for the first channel of each path, (design spec)
nt=16; %total number of channels (design spec)
n1=8; % numebers of channles in on path (design param)
np=nt/n1; % number of pathes
dlamCh=2*(10^(-9)); %(design spec)
dlamTot=(nt-1)*dlamCh;
lam(1,1)=lam1;
nch =0;
for i=1:1:np
    for j=1:1:n1
        nch=nch+1;
        lam(i,j)=lam(1,1)+(nch-1)*dlamCh;
    end
end
lam(1,8);
fsr=lam(2,1)-lam(1,1);
m=lam(1,1)/fsr;
m=vpa(int64(m))-2; % design param, number 2 is just choesn intuitively and trial and error.
neff = 1.6532; %(design spec/param)
BW = 25*(10^(-10)); %(design spec)
fsr=lam(1,1)/m;

% creating data output structur3 with time stamp
dateStamp = datestr(now,'ddmmmyy_HHMM');
designFilesDir='designSpecs';
mkdir(designFilesDir);
designFile=strcat('./',designFilesDir,'/designSpecPara_',dateStamp,'.txt');
outID=fopen(designFile,'a+');

% Calling nchfilter3.m for each path to calculate design parameters for
% each path.
\end{verbatim}

        \color{lightgray} \begin{verbatim}Warning: Directory already exists. 
\end{verbatim} \color{black}
    

\subsection*{nfilter3.m. Uses Straight forward theory and design specs}



\subsection*{params}

\begin{enumerate}
\setlength{\itemsep}{-1ex}
   \item neff: material/structure dependent.
   \item m: wavelenght number, m=Lam/FSR.
\end{enumerate}


\subsection*{Spec}

\begin{enumerate}
\setlength{\itemsep}{-1ex}
   \item n:    Number of channels.
   \item lam11:    Wavelength window = [lam11, lam11+FSR1]
   \item BW; This will set the minimun cross-couplingvalue, which is related to the gap between the waveguide and the ring.
   \item The limits on losses will put limits on the ring sizes. I haven't considred this effect here for now.
   \item FSR: is defined based on wavelength diff between channles and the fist wavelength. (FSR \ensuremath{>} n*lam11, where n is the number of channels in one path.)
\end{enumerate}


\subsection*{Design constraints:}

\begin{enumerate}
\setlength{\itemsep}{-1ex}
   \item Avoid resonant spliting; which gives us the maximun value for cross-coupling coeff.
   \item Lower crosstalk which calls for high-Q filter, i.e. higher ordef filters.
\end{enumerate}

\begin{verbatim}Note:
Spec is to have the 8 channles in the first channel's first FSR interval
Meaning the channels window is [lam1 lam1+fsr1]
n=8 for an 8-ch filter
We chose to have equally spced channeles with wavelength difference as dlam1\end{verbatim}
    

\subsection*{Function's Inputs}

\begin{enumerate}
\setlength{\itemsep}{-1ex}
   \item designFile:   which is the full filename (including the path) of output file. This output file contained required netlist params, design spec and some output info about the channel wavelength, FSRs, FWHMs.
   \item neff:         Index of reflection which is choosen.
   \item lam11:        Wavelength for the first channel (comes from desin spec)
   \item dlam1:        Space between channels (comes from desin spec). We have chosen to work on equally spaced channels. It is simple to accomodate not-equally-spaced channels if needed.
   \item m:             Wavelength number, It is chosen based on wavelength window size. (The calculation is done in parallerFilter.m.
   \item n:             Number of channels which is given as design spec.
\end{enumerate}


\subsection*{Function's Outputs:}

\begin{enumerate}
\setlength{\itemsep}{-1ex}
   \item Ring sizes,
   \item cross-coupling coef.
   \item Will also print the calculated, channel wavelenght, FSRs and ... at location "./designSpecs/designSpecPara\_\ensuremath{<}TIME STAMP\ensuremath{>}.txt"
\end{enumerate}
\begin{verbatim}
for i=1:1:np
    fprintf(outID,'\n\n************     PATH %d    ************\n\n',i);
    nchfilter3(designFile,neff,lam(i,1),dlamCh,eval(m),n1, BW)
end
\end{verbatim}


\subsection*{Next steps}

\begin{itemize}
\setlength{\itemsep}{-1ex}
   \item Next is updating the netlist files with the calculated design params at     \textit{./designSpecs/designSpecPara\ensuremath{<}TIME STAMP\ensuremath{>}.txt} .
   \item Then run the netlist files.
   \item Then use the ploting scripts
   \item * chParallerChecks.m which will call PlotAllCrossTalkImvCh8.m
\end{itemize}



\end{document}
    
